% Options for packages loaded elsewhere
\PassOptionsToPackage{unicode}{hyperref}
\PassOptionsToPackage{hyphens}{url}
%
\documentclass[
]{book}
\usepackage{lmodern}
\usepackage{amssymb,amsmath}
\usepackage{ifxetex,ifluatex}
\ifnum 0\ifxetex 1\fi\ifluatex 1\fi=0 % if pdftex
  \usepackage[T1]{fontenc}
  \usepackage[utf8]{inputenc}
  \usepackage{textcomp} % provide euro and other symbols
\else % if luatex or xetex
  \usepackage{unicode-math}
  \defaultfontfeatures{Scale=MatchLowercase}
  \defaultfontfeatures[\rmfamily]{Ligatures=TeX,Scale=1}
\fi
% Use upquote if available, for straight quotes in verbatim environments
\IfFileExists{upquote.sty}{\usepackage{upquote}}{}
\IfFileExists{microtype.sty}{% use microtype if available
  \usepackage[]{microtype}
  \UseMicrotypeSet[protrusion]{basicmath} % disable protrusion for tt fonts
}{}
\makeatletter
\@ifundefined{KOMAClassName}{% if non-KOMA class
  \IfFileExists{parskip.sty}{%
    \usepackage{parskip}
  }{% else
    \setlength{\parindent}{0pt}
    \setlength{\parskip}{6pt plus 2pt minus 1pt}}
}{% if KOMA class
  \KOMAoptions{parskip=half}}
\makeatother
\usepackage{xcolor}
\IfFileExists{xurl.sty}{\usepackage{xurl}}{} % add URL line breaks if available
\IfFileExists{bookmark.sty}{\usepackage{bookmark}}{\usepackage{hyperref}}
\hypersetup{
  pdftitle={Recipes by Anna \& Ivan},
  pdfauthor={(1+V)Anya},
  hidelinks,
  pdfcreator={LaTeX via pandoc}}
\urlstyle{same} % disable monospaced font for URLs
\usepackage{longtable,booktabs}
% Correct order of tables after \paragraph or \subparagraph
\usepackage{etoolbox}
\makeatletter
\patchcmd\longtable{\par}{\if@noskipsec\mbox{}\fi\par}{}{}
\makeatother
% Allow footnotes in longtable head/foot
\IfFileExists{footnotehyper.sty}{\usepackage{footnotehyper}}{\usepackage{footnote}}
\makesavenoteenv{longtable}
\usepackage{graphicx}
\makeatletter
\def\maxwidth{\ifdim\Gin@nat@width>\linewidth\linewidth\else\Gin@nat@width\fi}
\def\maxheight{\ifdim\Gin@nat@height>\textheight\textheight\else\Gin@nat@height\fi}
\makeatother
% Scale images if necessary, so that they will not overflow the page
% margins by default, and it is still possible to overwrite the defaults
% using explicit options in \includegraphics[width, height, ...]{}
\setkeys{Gin}{width=\maxwidth,height=\maxheight,keepaspectratio}
% Set default figure placement to htbp
\makeatletter
\def\fps@figure{htbp}
\makeatother
\setlength{\emergencystretch}{3em} % prevent overfull lines
\providecommand{\tightlist}{%
  \setlength{\itemsep}{0pt}\setlength{\parskip}{0pt}}
\setcounter{secnumdepth}{5}
\usepackage{booktabs}
\usepackage{amsthm}
\makeatletter
\def\thm@space@setup{%
  \thm@preskip=8pt plus 2pt minus 4pt
  \thm@postskip=\thm@preskip
}
\makeatother
\usepackage[]{natbib}
\bibliographystyle{apalike}

\title{Recipes by Anna \& Ivan}
\author{(1+V)Anya}
\date{2020-11-14}

\begin{document}
\maketitle

{
\setcounter{tocdepth}{1}
\tableofcontents
}
\hypertarget{baking}{%
\chapter{Baking}\label{baking}}

\hypertarget{macaron-endless-vanilla}{%
\section{Macaron ``Endless Vanilla''}\label{macaron-endless-vanilla}}

For biscuits:

\begin{itemize}
\tightlist
\item
  150 g almond flour, sifted twice
\item
  150 g powdered sugar
\item
  55 g egg whites, aged (from about 1.5 eggs)
\item
  1 1/2 vanilla pod
\item
  150 g icing sugar
\item
  37 g mineral water (without gases)
\item
  55 g egg whites, aged
\end{itemize}

For vanilla ganache:

\begin{itemize}
\tightlist
\item
  200 g heavy cream
\item
  1 vanilla pod from Mexico
\item
  1 vanilla pod from Madagascar
\item
  1 vanilla pod from Tahiti
\item
  220 grams of white chocolate (Valrhona preferred, but not critical)
\end{itemize}

Eggs:
You have noticed that I have the term ``aged'' proteins. This means that the day before you cook macarons, the egg whites need to be removed from the refrigerator, covered with cling film and allowed to stand at room temperature all this time. Such egg whites will become more liquid, they seem to decompose (the structure of the protein is destroyed). As a result, they become more voluminous when whipped.

Ganache:
Cut the vanilla pods in half. Use a knife to remove all the seeds. Add them along with the pods to a saucepan and pour the cream.
Bring to a boil over medium heat. Remove from heat, cover and let sit for 30 minutes.
Break the chocolate into pieces and melt in a water bath.
Heat the cream again. Remove the vanilla pods from the saucepan and gradually pour the vanilla cream into the chocolate in a thin stream, stirring constantly with a whisk so that there are no lumps.
Cover with cling film and refrigerate overnight.

Macarons:
Line a baking sheet with parchment paper. Prepare a cooking syringe or a piping bag with a straight round nozzle.

Macarons are made with a diameter of 3-4 cm. To begin with, while your hand is not trained, you can draw stencils for yourself on parchment paper, on the back side. In France, for example, you can buy baking paper with a stencil already marked or a silicone mat.

Sift almond flour. If large pieces of nuts remain in the sieve, grind them over again or leave them on a biscuit, cake, pie.
Sift almond flour and powdered sugar several times into a bowl.

Remove the seeds from the vanilla pods and add to the almond mixture. Do not stir. Add the first batch of egg whites (55 g) - do not stir.

In a small saucepan, combine water and second powdered sugar and bring to a boil until it reaches 118'C.
Whisk the second 55 g of egg whites until firm.

Gradually pour in the hot syrup in a thin stream. Continue whisking until the mixture is cool and shiny, thick and smooth.

When you lift the whisk, a mass should remain on its tip and not fall. A ``nose'' of proteins in a bowl, do not stand straight, but ``fall''. That said, if you turn your bowl of whites upside down, nothing should fall or leak.

Add the resulting Italian meringue (egg whites with hot syrup) to the first mixture.
Stir gently with a spatula while rotating the bowl counterclockwise with the other hand.

The mass should turn out to be homogeneous, soft, pliable. If you raise the spatula, then the drops of dough that will fall down should slowly spread, and not keep their shape. Do not be afraid to interfere - if the dough is poorly mixed, the pasta you have planted will have ``tails'' on the surface. Otherwise, they will slowly spread and take an even shape.

Place the dough in a syringe or bag and place in even circles on the lined paper.

And leave them to stand at room temperature for 1 hour. This is a very important stage of cooking - a light film forms on the surface of the meringue, due to which they do not crack during baking and a beautiful ``skirt'' is formed at the bottom.
Heat the oven to 175C.
Bake the cookies for 12-15 minutes. During cooking, 2 times (at the 8th and 10th minutes) very quickly open and close the oven, being careful not to knock.

Take out the finished cookies, grab the edges of the paper and transfer it along with the macarons onto a flat surface. Let cool completely.

A properly baked cookie will easily come off the paper surface.
When the cookies have cooled, fill a syringe or bag with ganache and place a small amount on one half, then close the other.

Refrigerate for 24 hours. Only cooked macaroons are never eaten - the halves will be tough and the whole cream will spread. They need to be given time to ``come up'', to brew, and then they will get exactly what they are so fond of: crisp crust, tender center and melting filling.

\hypertarget{upside-down-cake-with-nuts-and-bananas}{%
\section{Upside down cake with nuts and bananas}\label{upside-down-cake-with-nuts-and-bananas}}

\begin{itemize}
\item
  130g butter
\item
  120g sugar
\item
  120g flour
\item
  50g ground roasted nuts
\item
  3 eggs
\item
  0.5 tsp baking powder
\item
  2 bananas
\item
  50g dark muscovado sugar
\item
  20g butter
\end{itemize}

cake pan 10x20cm
oven 170C

First, brush the sides of the mold with softened butter and sprinkle with flour. Combine dark sugar and 20g butter and place on the bottom of the mold.
Prepare the dough. All foods should be at room temperature. Beat butter with sugar.
Add two eggs, one at a time, whisking each time until smooth at maximum mixer speed.
Add the nuts, add the remaining egg and beat the mixture again with a mixer.
Add flour and baking powder, stir on low speed until smooth.
Cut each banana into three pieces lengthwise. Place the cooked bananas in a mold on top of the sugar mixture.
Place the dough on top and tap well with the mold on the table.
Bake at 170C for 50 minutes.
Remove and cut the top off by sliding the knife along the edge of the pan and flip the hot cupcake onto a plate.

\hypertarget{spicy-donutsmuffins-from-the-oven-with-apples}{%
\section{Spicy donuts/muffins from the oven with apples}\label{spicy-donutsmuffins-from-the-oven-with-apples}}

\begin{itemize}
\tightlist
\item
  Flour - 140 g
\item
  Cinnamon - 1 tsp
\item
  Ginger - 1 tsp
\item
  Baking powder - 1 tsp.
\item
  Sugar - 140 g
\item
  Condensed milk - 100 g
\item
  Egg - 1 piece
\item
  Butter - 25 g
\item
  Apple
\end{itemize}

In a cup, combine flour (140 g), cinnamon and dried ginger (1 tsp each, I would not add fresh ginger), baking powder (also 1 tsp). And sugar (140 g). Stir the whole mass very well and carefully, because we need the baking powder, and the spices, to be distributed very evenly in the future dough. Of course, you can refuse cinnamon and ginger, or replace / supplement them with other spices - all at your discretion. Further condensed milk (100 g, you can replace with fat sour cream).

Melted butter (25 g). One egg. And at the very end, grate the apple (110-120 g). Traditionally, I take Granny Smith, because he can be found anywhere in Russia. Here you can play with filling. Add chopped nuts, oatmeal, dried berries. Mix the finished dough well and transfer to a bag.

This will make it most convenient for you to use it. Place the dough into molds with the expectation that it will rise by about one and a half times. Obviously, these are not real donuts (they are only deep-fried), so you can safely use any small-sized molds - make cupcakes, mini cupcakes and other options.

Bake at 180 degrees (top-bottom) until tender. Check with a skewer.

\hypertarget{carrot-cake}{%
\section{Carrot cake}\label{carrot-cake}}

For cakes:
* 360 ml (1.5 cups) vegetable oil
* 200 g (4 pcs) eggs
* 300 g (2 cups) flour
* 460 g (2 cups) sugar
* 8 g (1 sachet) baking powder
* 15 g (2 tsp) baking soda
* 3 g (1/2 tsp) salt
* 12 grams (1.5 tsp) cinnamon
* 5 g (0.5 tsp) nutmeg
* 5 g (0.5 tsp) cloves
* 500 g of carrots, grated on a fine grater
* 200 g (1 cup) nuts
For the cream:
* 800 g of curd cheese
* 120 g icing sugar
* 150 ml cream 33\%

First you need to combine vegetable oil with sugar. To do this, pour oil into the mixer bowl and add sugar. Beat at the maximum speed of the mixer for about 7 minutes, until the components are combined.

While the butter and sugar are beating, combine all dry ingredients in a separate bowl: sifted flour, salt, baking powder, soda, cinnamon, cloves, nutmeg. Stir well with a whisk until evenly spread.

When the butter and sugar have combined into a single emulsion, add the eggs one at a time, letting each egg dissolve completely (stir for 4-5 minutes). After all the eggs are added, you can add the dry mixture to the liquid one.

It is better to knead with a silicone spatula, since the mixture is not sticky and lumps are not formed during stirring. The result is a smooth, thick, but flowing dough.

It remains to add chopped nuts and grated carrots. It will be difficult to stir at first, but after a couple of minutes these ingredients will be evenly distributed throughout the mixture.

Fasten the bottom of the molds with baking paper and wrap in foil. Divide the mixture into three equal portions and place in separate baking tins. It is convenient to do this on a scale, for this you need to know the weight of the container in which all the ingredients were mixed. Spread the dough evenly with a spatula. Put in an oven preheated to 175 degrees, bake until tender for about 35 minutes.

Remove ready-made biscuits from the oven and let cool. Carefully cut out of the molds with a knife, remove the parchment from the bottom of the biscuits. We have got aromatic, juicy, porous biscuits. wrap in cling film, it is better to separate each cake, put in the refrigerator for several hours.

After the biscuits have cooled completely, you can start assembling the cake.

The recipe for the curd cream used for this cake can be found here.

Apply a small amount of cream to the cake base so that the cakes do not slip during assembly. Soak the cake with caramel (we took cherry), but you can not use the impregnation, since the biscuits are so juicy and bright.
Put some of the cream on the soaked biscuit, spread in an even layer, smooth the edges by going over the side of the cake with a spatula, trying to keep it perpendicular to the substrate.

Put the next cake and repeat the manipulation of the biscuit impregnation and cream application.

We complete the assembly of the cake with the last third biscuit, remembering to soak it. Apply the remaining cream to the top of the cake and the sides, closing all the small gaps.

We have got a delicious, tender, even cake in a ``half-naked'' coating. We leave it for an hour and a half in the refrigerator, so that the cream stiffens a little, and it is convenient to cut the cake. Decorate as desired (in our case, there were gingerbread cookies, some crushed nuts and fresh berries).

\hypertarget{strawberry-cake}{%
\section{Strawberry cake}\label{strawberry-cake}}

For cake
* 225 g butter
* 225 g sugar
* 4 eggs
* 225 g self-rising flour (or 220 g plain flour + 1 tablespoon baking powder)
* 3 tbsp milk

For filling
* 300 ml heavy cream
* 300 g strawberries
* 100 ml strawberry jam

Beat the softened butter with sugar. Add the eggs one at a time, beating without stopping. The mass must be airy. Gently put in the flour, mix the dough with a spoon, drawing an eight: this way the mixture loses a minimum of air. Add milk. We bake for 1 hour in a greased form at a temperature of 160 degrees Celsius.

At this time, we rest. The cream is resting in the refrigerator. Cold cream is easier to whip.

We take the cake out of the oven, let it cool in the form for 15 minutes. We take it out of the mold and leave it to cool again. At this time, cut the strawberries into slices, whip the cream. Add strawberry jam to the whipped cream and mix slightly. Precisely slightly, so that the cream turns out beautiful streaks of jam, and not a solid pink mass.

Then everything is intuitively clear: cut the cake lengthwise into three parts, put half the cream and half the strawberries on the first layer. Put in the second layer, lay out the remaining cream and strawberries. Cover with a third layer and sprinkle with a spoonful of powdered sugar. We decorate (if we have time before it is eaten) with fresh strawberries.

\hypertarget{cake-with-ricotta-and-pears}{%
\section{Cake with ricotta and pears}\label{cake-with-ricotta-and-pears}}

\begin{itemize}
\tightlist
\item
  4-5 pears
\item
  Dark sugar 2tsp
\item
  140 g butter
\item
  100 g sugar
\item
  1 egg
\item
  140 g ricota
\item
  zest of 1 lemon
\item
  200 g flour (can be less)
\item
  10 g baking powder
\end{itemize}

Put paper in a split form, grease with oil and sprinkle with brown oil. Then lay out the chopped pear in slices.
Beat butter and sugar until white, add egg, ricotta and zest.
Sift flour (do not add all the flour at once! How much it will take), baking powder and vanilla sugar, mix everything until smooth.
Pour the dough over the pears and put them in the oven for 50-60 minutes at 180C.

\hypertarget{cottage-cheese-cake-with-raspberries-or-apricots}{%
\section{Cottage cheese cake with raspberries or apricots}\label{cottage-cheese-cake-with-raspberries-or-apricots}}

\begin{itemize}
\tightlist
\item
  2 eggs
\item
  75 g sugar
\item
  Lemon zest
\item
  200 g cottage cheese
\item
  40 g starch
\item
  1 tablespoon ground almonds
\item
  200 g raspberries (can be frozen)
\end{itemize}

If the raspberries are frozen, remove them from the freezer at the beginning of the pie.
Beat eggs with sugar for several minutes until a fluffy light mass is obtained.
Add finely grated lemon zest.
Rub the cottage cheese through a sieve and hang with beaten eggs, add starch.
Place the dough in a greased dish and sprinkle with ground almonds or flour, spread the raspberries on top. Bake for 45-50 minutes at 180C.

\hypertarget{rudolphs-cookies-for-christmas-and-new-year}{%
\section{Rudolph's cookies for Christmas and New Year}\label{rudolphs-cookies-for-christmas-and-new-year}}

\begin{itemize}
\tightlist
\item
  280 g flour
\item
  100 g nut flour
\item
  130 g sugar
\item
  A bit of salt
\item
  200 g cold butter
\item
  2 egg yolks
\item
  Vanilla
\item
  1 tbsp of lemon juice
\item
  Lemon zest
\end{itemize}

Mix all together into crumbles. Make a ball, split into several parts and put in fridge for 30 min.
Form biscits and bake on 175C, for 12-13 min.

\hypertarget{swedish-chocolate-cake}{%
\section{Swedish chocolate cake}\label{swedish-chocolate-cake}}

\begin{itemize}
\tightlist
\item
  135 g butter
\item
  50 g cacao
\item
  180 g sugar
\item
  110 g flour
\item
  3 eggs
\end{itemize}

Melt butter, add everything step by step.
Put in a baking form and bake on 180C until the first cracks.

\hypertarget{scones}{%
\section{Scones}\label{scones}}

for 20 scones
* 610 g flour T45
* 610 g flour T55
* 60 g baking powder
* 280 g powder sugar
* 4 g salt
* 60 g trimoline (inverted sugar)
* 280 g butter
* 400 g milk
* 240 g buttermilk
* 120 g white raisins

Mix all dry elements with butter in a standing mixer with a mixing tool.
Change a mixing tool for a dough hook when butter disolves in dry ingredients. Add milk and buttermilk, mix for 3 min on Speed1, then 2 min on Speed2.
Split the dough into two

\hypertarget{apple-and-cinnamon-rolls}{%
\section{Apple and cinnamon rolls}\label{apple-and-cinnamon-rolls}}

\begin{itemize}
\tightlist
\item
  240+50 g flour
\item
  65 g warm milk
\item
  40 g butter
\item
  30 g sugar
\item
  1 egg yolk
\item
  1/2 sachet of yeast (or 10g of fresh yeast)
\item
  60 g water
\end{itemize}

For filling:
* 3 apples
* 60 g sugar
* 50 g butter
* 50 g blond raisins
* 1 tsp grounded cinnammon

For cream:
* 100 g sugar
* 100 g Philadelphia

In a mixing bowl (e.g., KitchenAid) mix yeast, water and warm milk, add 240 g of flour, sugar, salt and egg yolk, and mix it for 10 minutes. Add butter and mix another 5 minutes. Cover the dough with a film and put to rest for 2 hours.

Peel apples, cut into slices and put in a saucepan with the butter, then add sugar, raisins and cinnammon. Cook for 15 minutes, often steering. Let it cool down.
Roll the dough into a rectangle shape on a dusted surface, spread the apple filling, roll it and cut into 9 buns. Cover a baking dish (25cm) with a baking paper and place all buns next to each other. Leave it rest for 1 hour.
Heat the oven up to 180C. Place buns there for 25 minutes.
Prepare the cream: whip cheese with sugar and leave it aside.
When buns are not hot, spread cream all over them. Enjoy!

\hypertarget{main-dishes}{%
\chapter{Main dishes}\label{main-dishes}}

\hypertarget{broccoli-baked-in-cream-with-cheese}{%
\section{Broccoli baked in cream with cheese}\label{broccoli-baked-in-cream-with-cheese}}

\begin{itemize}
\tightlist
\item
  1 head of broccoli
\item
  1 egg
\item
  100 ml cream 10\%
\item
  nutmeg (on the tip of a knife)
\item
  a handful of cheese
\end{itemize}

Divide the broccoli into inflorescences and boil in boiling water for 3 minutes. Not more! This is enough for the initial heat treatment, while the broccoli remains crispy and tasty.Combine egg, cream and nutmeg.
Put broccoli in a mold, pour over the filling, sprinkle with grated cheese on top.
Bake in the oven for 15 minutes at 170C.

\hypertarget{adjarian-khachapuri-bread-with-cheese---acharuli}{%
\section{Adjarian Khachapuri (bread with cheese) - Acharuli}\label{adjarian-khachapuri-bread-with-cheese---acharuli}}

For 2-4 boats (depending on size)
Ingredients:

\begin{itemize}
\tightlist
\item
  380-400 gr flour
\item
  200 ml warm water
\item
  100 ml milk
\item
  40 ml vegetable oil
\item
  1 tsp dry yeast
\item
  1 tsp salt
\item
  2 tsp sugar
\end{itemize}

For filling:
* 1 egg
* 200 gr suluguni (mozarella)
* 400 gr Imeretian cheese (feta + cheddar/parmezan)
* 50-70 ml milk or water
* 1.5 tbsp flour
* 2-4 yolks (depending on the number of boats)
* 50 gr butter (optional)

Preparation:
1. Dissolve the yeast with sugar in warm water, leave for 5 minutes. Add milk, mix and gradually add 200 g of flour, knead by hand or with a food processor until smooth, then gradually add the rest of the flour in the same way. (The dough should be collected in a ball, but with it is rather sticky). Add vegetable oil in parts and continue to knead. (The dough will become more obedient and smooth, slightly sticking to the fingers.) Then cover the dough with cling film or a towel, put in a warm place and let it increase at least twice. (You can also put it in the refrigerator for night if you are not going to bake everything at once)
2.Grate Imeretian cheese (you can replace it with Adyghe or vats, or at least feta). Add one egg, milk (water), flour, salt if necessary, if the cheese is not salted at all, mix everything into a homogeneous mass, it should be not dry, but like porridge.
3. Separately grate the suluguni on a coarse grater (can be replaced with mozzarella).
4. Divide the matched dough into 2-4 parts, depending on the size of the boat (khachapuri), form balls and leave for 10 minutes, covered with cling film or a towel. The dough will again become soft, pliable and more fluffy. Hands and work surface should be dusted with flour to work comfortably with the dough.
5.I recommend forming the boats directly on the parchment to make it easier to transfer to a baking sheet.

Forming No.~1:

With your fingers open the ball into a cake, while pressing only in the center of the cake, so that the edges remain thick and the center is thin \ldots{} Then take the cake in your hands and twist, stretching the dough (like on a pizza). Put again on the work surface, stretch the circle in a light oval, bring two edges of the dough to the middle, just pull the other two edges, lengthening the resulting shape of the dough, then connect. It turns out the shape of a ``canoe'' in which the edges are stuck together. Then open the center with your fingers, stretching the dough in width, forming the shape of a boat with sides .Put the cheese mass in the center of the boat, distribute and pour grated suluguni on top.

Forming No.~2

Open a uniform flat cake with your fingers, pick it up and stretch it slightly in all directions to make a thinner middle. Put the cheese mass on the cake, spread over the entire surface, leaving only the edges free. Wrap the sides of the cake, grabbing the cheese part. (As if you are wrapping a roll), while glue only the ends, creating the ends of the boat. Pour grated suluguni over the open cheese mass.

\begin{enumerate}
\def\labelenumi{\arabic{enumi}.}
\setcounter{enumi}{5}
\tightlist
\item
  In a preheated oven (220 degrees) and a baking sheet, transfer the formed boats, bake for 15 minutes, until golden brown.
\item
  Remove the boats from the oven. Optionally, you can remove the excess dough from the khachapuri formed according to method No.~1. Separate the cheese part from the dough with a fork, pry it under the side and gently move the hand, remove the dough. Push cheese into the empty sides, put the yolk on top , then back into the oven for 1-2 minutes. And on the khachapuri formed according to the method No.~2, simply put the yolk on top and again in the oven for 1-2 minutes. If desired, after the oven, coat the edges of the dough with butter and serve hot with a piece oil next to szheltkom.
\end{enumerate}

\hypertarget{baked-aubergines-with-mozzarella-cheese}{%
\section{Baked aubergines with mozzarella cheese}\label{baked-aubergines-with-mozzarella-cheese}}

2 servings

\begin{itemize}
\tightlist
\item
  1 medium aubergines
\item
  1 scoop of mozzarella
\item
  50 g parmesan
\item
  5-6 medium tomatoes (or a can of chopped tomatoes)
\item
  1 clove of garlic
\item
  several sprigs of basil
\item
  olive oil
\item
  flour
\item
  salt
\item
  black pepper
\end{itemize}

Cut the aubergines into circles that are not too thick (about 0.5 cm), lightly add salt and place in a colander for about half an hour: during this time, as they say, excessively bitter juices will come out of them. Also slice the mozzarella thinly and grate the Parmesan cheese. Peel tomatoes (how to peel tomatoes) and cut into small pieces. Fry chopped garlic and basil stalk in a little olive oil, add chopped tomatoes and basil leaves, season with salt and pepper and simmer for 10-15 minutes over medium heat until the sauce thickens. Remove from heat and strain through a sieve or grind in a blender.

Rinse the aubergines, dry quickly, roll in flour and lightly fry in olive oil (aubergines tend to absorb oil like a sponge, so add some oil if necessary before loading the pan with the next portion). Transfer the fried aubergines to napkins to absorb the excess oil. Everything is ready to assemble our ``tower'': take a baking dish (best glass or ceramic), put a spoonful of tomato sauce on the bottom, place a slice of eggplant on top, put a slice of mozzarella on top and sprinkle with Parmesan. The next ``floor'' is tomato sauce, eggplant, cheese, and so on until you build as many towers as you plan to make eaters happy with this delicious dish. Bake the aubergines in an oven preheated to 220 degrees until the cheese melts on top into a golden crust (this will take 15-20 minutes), and serve immediately.

\hypertarget{crispy-greek-style-pie}{%
\section{Crispy Greek-style pie}\label{crispy-greek-style-pie}}

\begin{itemize}
\tightlist
\item
  200g bag spinach leaves
\item
  175g jar sundried tomato in oil
\item
  100g feta cheese, crumbled
\item
  2 eggs
\item
  0.5 250g pack filo pastry
\end{itemize}

Put the spinach into a large pan. Pour over a couple tbsp water, then cook until just wilted. Tip into a sieve, leave to cool a little, then squeeze out any excess water and roughly chop. Roughly chop the tomatoes and put into a bowl along with the spinach, feta and eggs. Mix well.

Carefully unroll the filo pastry. Cover with some damp sheets of kitchen paper to stop it drying out. Take a sheet of pastry and brush liberally with some of the sundried tomato oil. Drape oil-side down in a 22cm loosebottomed cake tin so that some of the pastry hangs over the side. Brush oil on another piece of pastry and place in the tin, just a little further round. Keep placing the pastry pieces in the tin until you have roughly three layers, then spoon over the filling. Pull the sides into the middle, scrunch up and make sure the filling is covered. Brush with a little more oil.

Heat oven to 180C/fan 160C/gas 4. Cook the pie for 30 mins until the pastry is crisp and golden brown. Remove from the cake tin, slice into wedges and serve with salad.

\hypertarget{lamb-stuffed-aubergines-with-moorish-spices-and-manchego-cheese}{%
\section{Lamb-stuffed aubergines with Moorish spices and Manchego cheese}\label{lamb-stuffed-aubergines-with-moorish-spices-and-manchego-cheese}}

\begin{itemize}
\tightlist
\item
  4 aubergines
\item
  6 tbsp olive oil
\item
  1 onion, chopped
\item
  4 garlic cloves, finely chopped
\item
  1 large red pepper, seeds removed, chopped
\item
  1½ tsp freshly ground cumin seeds
\item
  1 tsp ground cinnamon
\item
  ½ tsp freshly grated nutmeg
\item
  1 tsp pimentón dulce (smoked sweet Spanish paprika)
\item
  large pinch of crushed dried chillies
\item
  500g/1lb 2oz lamb mince
\item
  6 tbsp tomato sauce (see Top recipe tip below)
\item
  100g/3½oz Manchego cheese, coarsely grated
\item
  salt and freshly ground black pepper
\end{itemize}

Preheat the oven to 200C/400F/Gas 6.
Cut each aubergine lengthways through the stalk, then score the flesh in a tight criss-cross pattern, taking the knife through the flesh down to the skin, but taking care not to cut through the skin. Place them side by side on a baking tray and drizzle each half with half a tablespoon of the oil, season with salt and bake for 30-40 minutes or until the flesh is soft and tender but not browned.
Meanwhile, heat the remaining two tablespoons of oil in a large non-stick frying pan. Add the onion, garlic, red pepper and spices and fry gently for 10 minutes. Add the lamb mince and fry for 3--4 minutes or until all the meat is lightly browned. Stir in the tomato sauce and simmer for five minutes.
Remove the aubergines from the oven and increase the temperature to 220C/425F/Gas 7. Carefully scoop most of the flesh out of the baked aubergine halves, leaving the skins with a layer of flesh about 1cm/½in thick. Stir the scooped-out flesh into the lamb mixture with half a teaspoon of salt and some pepper to taste. Spoon the mixture into each aubergine shell and sprinkle with the grated cheese. Bake in the oven for 8--10 minutes, or until the cheese is bubbling and golden-brown.
\#\# Simple baked lasagne

\begin{itemize}
\item
  2 carrots , peeled
\item
  2 onions , peeled
\item
  2 cloves of garlic , peeled
\item
  2 sticks of celery , trimmed
\item
  olive oil
\item
  2 rashers of higher-welfare smoked streaky bacon
\item
  ½ a bunch of fresh thyme
\item
  500 g quality beef mince
\item
  a good splash of red wine
\item
  1 gluten-free beef stock cube , preferably organic
\item
  2 x 400 g tins of plum tomatoes
\item
  sea salt
\item
  freshly ground black pepper
\item
  1 x gluten-free pasta dough
\item
  3 anchovy fillets , from sustainable sources
\item
  500 ml crème fraîche
\item
  2 handfuls of freshly grated Parmesan cheese
\item
  milk , optional
\end{itemize}

To make the Bolognese sauce, finely chop the carrots, onions, garlic and celery and add to a large, wide pan over a medium heat with a drizzle of olive oil. Roughly chop and add the bacon, then pick in the thyme leaves and cook for 5 to 10 minutes, or until softened and lightly golden.
Turn the heat up slightly, then stir in the beef mince, breaking it up with a wooden spoon. Cook for around 5 minutes, or until browned all over. Add the wine and crumble in the stock cube, stirring continuously until the liquid has completely reduced. Stir in the tomatoes and 1 tin's worth of water and bring to the boil. Reduce to a simmer, cover and cook for around 1 hour, then remove the lid and continue cooking for 30 minutes, or until thickened and reduced.
Meanwhile, preheat the oven to 180ºC/350ºF/gas 4.
For the white sauce, finely chop the anchovies, then mix with the crème fraîche and a handful of Parmesan, then season with salt and pepper -- you may need to loosen the mixture with a little milk.
Cut the sheets of pasta into rectangles (roughly 10cm x 15cm).
Spoon one-third of the Bolognese sauce into an ovenproof dish (roughly 25cm x 30cm). Layer over one-third of the lasagne sheets and top with one-third of the béchamel sauce. Repeat with the remaining ingredients until you have three layers in total, finishing with a final layer of béchamel. Grate over the remaining Parmesan and drizzle with olive oil, then cover with tin foil. Place in the hot oven for around 20 minutes, remove the foil and continue cooking for around 30 minutes, or until golden and bubbling. Serve with a nice, crisp salad.
\#\# Kung Pao chicken

\begin{itemize}
\tightlist
\item
  1 tablespoon Szechuan peppercorns
\item
  2½ tablespoons cornflour
\item
  4 skinless higher-welfare chicken thighs , (350g)
\item
  groundnut oil , or vegetable oil
\item
  4 cloves of garlic
\item
  5 cm piece of ginger
\item
  2 spring onions
\item
  6 dried red chillies
\item
  2 tablespoons low-salt soy sauce
\item
  ½ tablespoon rice wine vinegar
\item
  1 heaped tablespoon runny honey
\item
  50 g unsalted peanuts
\item
  1 punnet of salad cress
\end{itemize}

Toast the Szechuan peppercorns in a dry frying pan until lightly golden. Transfer to a pestle and mortar, grind to a fine powder, then sieve into a large bowl, discarding any large, tough bits.
Add 2 tablespoons of cornflour and stir to combine. Chop the chicken into bite-sized chunks, then toss in the cornflour mixture to coat.
Pour 2cm of oil into a large non-stick frying pan over a medium heat, add the chicken and fry for 7 to 8 minutes, or until golden and cooked through.
Meanwhile, peel and finely slice the garlic and ginger, then trim and finely slice the spring onions.
Using a slotted spoon, remove the chicken to a double layer of kitchen paper to drain. Carefully remove and discard most of the oil, leaving about 2 tablespoons in the pan, then return to a medium heat.
Add the garlic and ginger and fry for 2 minutes, or until lightly golden, then stir in the spring onions and whole chillies and fry for 1 further minute.
Meanwhile, combine ½ tablespoon of cornflour and 2 tablespoons of water. Mix in the soy, vinegar and honey, then pour the mixture into the pan. Bring to the boil and simmer for a few minutes, or until slightly thickened.
Lightly bash and add the peanuts, stir in the chicken, then toss well until warmed through. Snip the cress over the ribbon salad, scatter the reserved coriander leaves over the chicken, then serve.
\#\# Turkey con chilli

\begin{itemize}
\tightlist
\item
  olive oil
\item
  2 red onions , peeled and roughly chopped
\item
  1 carrot , peeled and roughly chopped
\item
  1 leek , trimmed and roughly chopped
\item
  1 red pepper , deseeded and roughly chopped
\item
  1 yellow pepper , deseeded and roughly chopped
\item
  1 fresh red chilli , deseeded and finely chopped
\item
  1 fresh green chilli , deseeded and finely chopped
\item
  1 bunch fresh coriander , stalks finely chopped, leaves picked
\item
  1 teaspoon ground cumin
\item
  1 heaped teaspoon smoked paprika
\item
  1 heaped teaspoon runny honey , optional
\item
  3 tablespoons white wine vinegar , optional
\item
  600 g turkey , leftover, shredded
\item
  sea salt
\item
  freshly ground black pepper
\item
  3 x 400 g tinned chopped tomatoes
\item
  400 g tinned butter beans or chickpeas , drained
\item
  2 limes , juice of
\item
  soured cream , to serve
  FOR THE GUACAMOLE
\item
  2 ripe avocados , peeled and destoned
\item
  2 tomatoes , halved
\item
  ¼ red onion , peeled
\item
  ½ clove garlic , peeled
\item
  1 fresh green chilli , deseeded
\item
  1 bunch fresh coriander
\item
  1 lime
\end{itemize}

Preheat the oven to 180ºC/350ºF/gas 4. Heat a few lugs of olive oil in a large casserole-type pan on a medium heat. Add the onions, carrot, leek, peppers and chillies, and cook, stirring occasionally, for about 5 minutes. Add the coriander stalks, cumin and paprika, and cook for another 10 minutes or so, stirring frequently until soft and delicious. Sometimes I like to add some honey and white wine vinegar at this point and let it cook for a couple of minutes. I find this adds a wonderful sheen and enhances the natural sweetness of the vegetables.
While that's happening, shred the turkey meat off your carcass and roughly chop it. Add a good pinch of salt and pepper to the pan of vegetables, then add the turkey and take the pan off the heat. Add the tomatoes and chickpeas or butter beans and stir everything together. Pop it in the hot oven to blip away for 2 hours. Check on it after an hour, and add a splash of water if it looks a bit dry.
While that's cooking, make your guacamole by blitzing one of your avocados in a food processor with the tomatoes, onion, garlic, chilli and coriander. Use a fork to mash the other avocado in a bowl so it's nice and chunky. Taste the mixture in the food processor and add salt and squeezes of lime juice until the taste is just right for you. Whiz up one more time then tip into the bowl with your chunky avocado and mix together.
Take the chilli out of the oven and scrape all the gnarly bits from the edge of the pan back into the chilli for added flavour. Squeeze in some lime juice, and stir through most of the coriander leaves. Have a taste to check the seasoning then serve with steamed basmati rice or tortillas, and a good dollop of soured cream and guacamole on top. Scatter over your remaining coriander leaves and some finely sliced fresh chilli if you fancy then get everyone to tuck in.

\hypertarget{prawn-chorizo-orzo}{%
\section{Prawn \& chorizo orzo}\label{prawn-chorizo-orzo}}

\begin{itemize}
\tightlist
\item
  2 cloves of garlic
\item
  200 g quality chorizo
\item
  0.5 a bunch of fresh basil (15g)
\item
  4 tablespoons olive oil
\item
  2 tablespoons sherry vinegar
\item
  400 ml passata
\item
  300 g orzo
\item
  200 g cherry tomatoes, on the vine
\item
  400 g large cooked peeled king prawns ,from sustainable sources
\end{itemize}

Preheat the oven to 180ºC/350ºF/gas 4.

Peel and finely chop the garlic, and slice the chorizo. Pick and finely chop the basil.

Heat half the oil in a heavy-based pan. Fry the garlic and chorizo for 3 minutes, then deglaze the pan with the vinegar.

Add the passata and 300ml of water, then the orzo. Bring to the boil, reduce the heat and simmer for 10 to 15 minutes, or until the orzo is al dente, stirring occasionally to prevent it sticking.

Spread the cherry tomatoes over a baking tray, drizzle with the rest of the oil and season. Roast for 10 minutes, or until soft.

Stir half the basil into the pasta, along with the prawns. Divide between bowls, top with the remaining basil and serve the roasted tomatoes alongside.

\hypertarget{chicken-kiev}{%
\section{Chicken kiev}\label{chicken-kiev}}

\begin{itemize}
\tightlist
\item
  4 rashers of smoked streaky bacon
\item
  olive oil
\item
  4 x 150 g skinless chicken breasts , (I got mine from the butcher with the bone in, but either way is fine)
\item
  3 tablespoons plain flour
\item
  2 large free-range eggs
\item
  150 g fresh breadcrumbs
\item
  sunflower oil
\item
  2 large handfuls of baby spinach , or rocket
\item
  2 lemons
\item
  BUTTER
\item
  4 cloves of garlic
\item
  ½ a bunch of fresh flat-leaf parsley (15g)
\item
  4 knobs of unsalted butter , (at room temperature)
\item
  1 pinch of cayenne pepper
\end{itemize}

Fry the bacon in a pan on a medium heat with a tiny drizzle of olive oil, until golden and crisp, then remove.
For the butter, peel the garlic, then finely chop with the parsley leaves and mix into the softened butter with the cayenne. Firm up in the fridge.
Working one-by-one on a board, stuff the chicken breasts. To do this, start by pulling back the loose fillet on the back of the breast -- put your knife in the opposite direction and slice to create a long pocket (for extra guidance, watch the how-to video below).
Open the pocket up with your fingers, cut the chilled butter into four and push one piece into the pocket, then crumble in a rasher of crispy bacon. Fold and seal back the chicken, completely covering the butter and giving you a nice neat parcel. Repeat with the 3 remaining breasts.
Preheat the oven to 180°C/350°F/gas 4.
Place the flour in one shallow bowl, whisk the eggs in another, and put the breadcrumbs and a pinch of seasoning into a third. Evenly coat each chicken breast in flour, then beaten egg, letting any excess drip off, and finally, turn them in the breadcrumbs, patting them on until evenly coated.
Shallow-fry in 2cm of sunflower oil on a medium to high heat for a couple of minutes on each side, or until lightly golden, then transfer to a tray and bake in the oven for 10 minutes, or until cooked through. You can bake them completely in the oven and skip the frying altogether, you just need to drizzle them with olive oil and bake for about 20 minutes -- they won't be as golden, but they'll be just as delicious.
Meanwhile, peel and roughly chop the potatoes and cook in a large pan of boiling salted water for 12 to 15 minutes, or until tender.
Chop up the broccoli and add it to the potatoes for the last 8 minutes. Drain and leave to steam dry, then return to the pan and mash with a knob of butter and a pinch of salt and pepper.
Divide the mash between your plates and place a Kiev on top of each portion. Lightly dress the spinach leaves or rocket in a little oil and lemon juice, then sprinkle over the top as a salady garnish. Serve with a wedge of lemon on the side.

\hypertarget{farfalle-with-carbonara-and-spring-peas}{%
\section{Farfalle with carbonara and spring peas}\label{farfalle-with-carbonara-and-spring-peas}}

\begin{itemize}
\tightlist
\item
  455 g farfalle
\item
  1 free-range egg
\item
  100 ml double cream
\item
  12 rashers of higher-welfare pancetta or smoked streaky bacon
\item
  3 handfuls of fresh podded or frozen peas
\item
  2 sprigs of fresh mint
\item
  2 handfuls of Parmesan cheese
\end{itemize}

First of all, bring a large pan of salted water to the boil, add the farfalle, and cook according to the packet instructions.
Whisk the egg in a bowl with the cream and season with sea salt and black pepper.
Roughly slice your pancetta or bacon and put it into a second pan over a medium heat and cook until crispy and golden.
When the farfalle is nearly cooked, add the peas for the last minute and a half. This way they will burst in your mouth and be lovely and sweet. When cooked, drain in a colander, saving a little of the cooking water.
Add the pasta to the pancetta. Pick and finely slice the mint leaves and stir in most of them. If the pan isn't big enough, mix it all together in a large warmed bowl.
Now you need to add the egg and cream mix to the pasta. What's important here is that you add it while the pasta is still hot. This way, the residual heat of the pasta will cook the eggs, but not so that they resemble scrambled eggs, as I've seen in some dodgy old restaurants on the motorway! The pasta will actually cook the egg enough to give you a silky smooth sauce. Toss together and loosen with a little of the reserved cooking water if necessary.
Season with salt and pepper, grate over the Parmesan and sprinkle over the rest of the mint leaves, and serve as soon as possible.

\hypertarget{taiwanese-3-cup-chicken}{%
\section{Taiwanese 3 Cup Chicken}\label{taiwanese-3-cup-chicken}}

\begin{itemize}
\tightlist
\item
  1 - 1 1/2 lb. chicken drumettes
\item
  15 - 20 cloves of garlic, peeled
\item
  1 small piece of ginger, sliced
\item
  fresh Thai basil leaves (red basil)
\end{itemize}

For the Sauce:
* 1/3 cup of soy sauce (low sodium)
* 1/3 cup rice wine
* 1/3 cup of Asian sesame oil
* 3 Tbsp. cane sugar
* 1 tsp. dried chili flakes (or fresh red chilis)
* 1/2 tsp. salt

Brown the chicken for a few minutes first before adding the sauce mixture.
Cook the sauce mixture and chicken for 15 - 20 min. or until almost done.
When almost done, add the fresh basil leaves and stir fry, then cover with a lid for 1 min.
Transfer to a serving bowl and serve hot.

\hypertarget{quiche-with-salmon-brocolli-and-blue-cheese}{%
\section{Quiche with salmon, brocolli and blue cheese}\label{quiche-with-salmon-brocolli-and-blue-cheese}}

\begin{itemize}
\item
  175 g flour
\item
  100g butter
\item
  2 tbsp cold water
\item
  1 egg yolk
\item
  Salt
\item
  300 g salmon
\item
  200 g brocolli
\item
  70 g blue cheese/ gorgonzola
\item
  3 eggs
\item
  250 g sour cream / fraiche cream or double cream
\item
  Salt, pepper, herbs, thyme
\end{itemize}

10-15 minutes before preparing the dough, put water and butter in the freezer.
Grate the butter on a coarse grater, stir with flour and add salt, water and yolk, quickly knead the dough for a short time until it can be collected into a ball. Wrap it in a cling film and cooled in a fridge for 30-40 minutes.
Cover the bottom of the mold with baking paper, roll out the dough between two sheets of parchment to a thickness of 3 mm, put in a mold with a diameter of 18-20 cm, make sides 4-5 cm high, prick with a fork, put in the freezer for 10 minutes.
Put a sheet of baking paper on the dough and load it with either dry rice or peas (to create a heavy weight to keep dough in place). Bake at 200C for 15 minutes with weight and 10 minutes without, until the base is lightly browned.
For pouring: mix eggs, sour cream, salt and pepper, herbs and crumble the cheese. Divide the broccoli into inflorescences and cut the fish into pieces.
When the base is ready, lay out the filling, pour the filling on top and sprinkle with some cheese (additional). Bake at 200 C for 40-50 minutes. Let cool for 15-20 minutes, you can cut it.

\hypertarget{starters}{%
\chapter{Starters}\label{starters}}

\hypertarget{vegetable-tartlets}{%
\section{Vegetable tartlets}\label{vegetable-tartlets}}

1 x 500 g block of ready-made puff pastry
plain flour , for dusting
4 teaspoons pesto
1 handful of mixed, ripe cherry tomatoes
8 asparagus spears
4 baby courgettes
2-3 jarred roasted peppers
½ a bunch of fresh basil
olive oil
8 black olives , optional
1 x 100 g ball mozzarella
20 g Parmesan cheese , optional

Turn on the oven to 200ºC/gas 6. Carefully cut the pastry block in half with a table knife. Wrap the other half and refrigerate or freeze for later.
Dust some flour onto a clean work surface and, using a rolling pin, roll out the pastry into a square, measuring 26cm x 26cm. Cut into 4 equal squares.
Place the pastry squares on a baking tray, leaving a space between each.
Using the back of a spoon, spread the centre of each square with pesto, but don't spread it onto the edges.
Squash the tomatoes into a large mixing bowl, then snap the asparagus spears into 3cm pieces. Keep the lovely pointy tips and a little of the stalk, but discard the end 3cm.
Using a speed peeler, carefully shred the courgettes into ribbons. Tear the roasted peppers into strips and add to the bowl.
Pick the basil leaves, reserving the pretty ones for later. Place the large ones in the mixing bowl.
Mix the vegetables together in the bowl, adding a splash of oil. Pile a little of the mixture on each pesto-smeared tart and top with two olives (if using).
Break up the mozzarella and place little bits on top of each tart -- this will make it gooey like a pizza. Grate over some Parmesan (if using).
Bake for 15 to 20 minutes, until the pastry is golden and the cheese is all bubbly.
Once the tarts are ready, allow to cool slightly. Sprinkle with the reserved basil leaves and serve with a nice salad for lunch.

\hypertarget{desserts}{%
\chapter{Desserts}\label{desserts}}

\hypertarget{tiramisu}{%
\section{Tiramisu}\label{tiramisu}}

\hypertarget{cheesecake}{%
\section{Cheesecake}\label{cheesecake}}

\hypertarget{panna-cotta}{%
\section{Panna Cotta}\label{panna-cotta}}

\includegraphics{images/IMG_2480.jpg}
For 10 portions:
* 1 l Double cream (33\%)
* 150 g sugar
* 6 sheets of gelatine
* 1 vanilla pod

\begin{itemize}
\tightlist
\item
  300 g strawberries
\item
  120 g sugar
\item
  1/2 lemon's juice
\item
  3 g pectin
\end{itemize}

Soak gelatin in cold water for a few minutes, squeeze. Slice the vanilla pod lengthwise and scrape out the seeds.
Place the cream, sugar, vanilla seeds and the pod in a saucepan. Dissolve the gelatin in them, stirring occasionally. When it comes to almost a boil, turn off and divide into jars through a sieve. Allow to cool and refrigerate for a couple of hours.
Place all the ingredients for the strawberry salsa in the pan and cook, stirring, until the sugar dissolves, and then until the strawberries soften. Place a few spoonfuls of warm salsa on top of the thickened panna cotta in jars. Refrigerate for 1 hour.

\hypertarget{special-occasions}{%
\chapter{Special occasions}\label{special-occasions}}

\hypertarget{roasted-duck}{%
\section{Roasted duck}\label{roasted-duck}}

\hypertarget{whole-baked-salmon}{%
\section{Whole baked salmon}\label{whole-baked-salmon}}

\hypertarget{breakfast-recipes}{%
\chapter{Breakfast recipes}\label{breakfast-recipes}}

\hypertarget{crepes}{%
\section{Crepes}\label{crepes}}

for two

\begin{itemize}
\tightlist
\item
  2 eggs
\item
  1 glass of water
\item
  1 glass of milk
\item
  1 cup self-rising flour
\item
  2 tbsp sugar (optional)
\item
  2 tbsp vegetable oil
\end{itemize}

Beat eggs with sugar (if using). Add a glass of hot water, whisking with a mixer, then a glass of flour and a glass of warm milk.
Add butter to the dough. Fry in a hot skillet without adding oil.

\hypertarget{true-belgian-waffles}{%
\section{True Belgian Waffles}\label{true-belgian-waffles}}

\begin{itemize}
\tightlist
\item
  2 cups all-purpose flour
\item
  0.75 cup sugar
\item
  3.5 teaspoons baking powder
\item
  2 large eggs, separated
\item
  1.5 cups whole milk
\item
  1 cup butter, melted
\item
  1 teaspoon vanilla extract
\item
  Sliced fresh strawberries or syrup
\end{itemize}

In a bowl, combine flour, sugar and baking powder. In another bowl, lightly beat egg yolks. Add milk, butter and vanilla; mix well. Stir into dry ingredients just until combined. Beat egg whites until stiff peaks form; fold into batter.

Bake in a preheated waffle iron according to manufacturer's directions until golden brown. Serve with strawberries or syrup.

\hypertarget{gluten-free-pancakeswaffles}{%
\section{Gluten-free pancakes/waffles}\label{gluten-free-pancakeswaffles}}

\begin{itemize}
\tightlist
\item
  130 g Yougurt
\item
  2 eggs
\item
  4 tbsp flour
\item
  0.5 tsp
\item
  Salt
\end{itemize}

Mix together and cook either on a pan or in a waffle maker without oil.

\hypertarget{cottage-cheese-pancakes}{%
\section{Cottage cheese pancakes}\label{cottage-cheese-pancakes}}

\begin{itemize}
\tightlist
\item
  200-250 g cottage cheese
\item
  1 egg
\item
  4-5 tbsp semolina/flour
\item
  vanilla
\end{itemize}

Mix all together and form small balls (around 6-8 items) and bake it until golden. Or cook it on a pan from both sides until golden on medium heat.

\end{document}
